\documentclass[a4paper,10pt]{article}
\usepackage[english]{babel}
\usepackage[hmargin=2.5cm,vmargin=3.5cm]{geometry}
\usepackage{microtype}
\pagestyle{empty}

\begin{document}

%Schottky-type should be changed to "inversion layer".  In the talk you can first explain Schottky solar cells then explain that inversion is a special case of this.
\section*{Shottky-type solar cells}
Schottky-type solar cells offer the possibility to attain light to electricity conversion efficiencies similar to ``standard'' solar cells, without the need for high-energy, high-cost \emph{p-n}-junction assembly from different types of silicon (\emph{Si}). The ultimate efficiency of the Schottky-type solar cell is largely dependant on the properties of the polymer film and its interface with \emph{Si}. Therefore, it is the aim of this project to optimise poly(ethylene dioxythiophene):poly(styrene sulfonate) (\emph{PEDOT:PSS}) film properties. %You need to add a sentence about how an inversion layer solar cell is formed, otherwise the sentence "the ultimate efficiency of the ... of the polymer film and its interface with Si" doesn't make so much sense.  "from different types of Si" should be "involving different types of dopants".  Not clear what "types" refers to if you're not a semiconductor scientist.
\section*{Optimising surfactant concentration}
This project's focus is on the optimal amount of fluoro-surfactant -- here Zonyl\textsuperscript{\textregistered} -- to be used in watery \emph{PDOT:PSS} solution also containg 5 \%\footnote{all weight-percentages are with respect to \emph{PDOT:PSS}} ethylene-glycole. Surfactant concentration has been varied in a set of experiments. The different solutions were spun-cast on glass plates, then analysed with four-point probe resistance measurements, transmission spectrometry as well as atomic force and optical microscopy. Solar cell devices were assembled from these films and \emph{n}-type-\emph{Si} in a lift-off float-on (\emph{LOFO}) process. Silver top contacts were created using electron beam evaporation and the devices' performance was tested with standard J-V measurements.
% I think something like "this project's focus is on optimizing the properties of the PEDOT:PSS film, deposited from an aqueous solution, to improve solar cell power efficiency." Of course you don't need to use those exact words, but something a little more general than "the optimal amount of fluoro-surfactant". Then you can add another sentence about focusing on the zonyl, and what properties of the film the zonyl can impact. you can leave out the footnote (this can be said in the talk).  Glycol (not glycole). Spectroscopy (not spectrometry), and thickness measurements, not atomic force microscopy (AFM is different than the dektak - similar concept, but smaller scale)
\section*{Experimental results}
On glass, maximum conductivity as well as maximum transmissivity was found for films containing 0.4\% Zonyl. Solar cell devices assembled using the \emph{LOFO}-process and films containing 0.4\% Zonyl showed the highest efficiency.\\
In a side project, the film's resistivity could be reduced by a factor of $\approx \, 4$ by introducing a thin film of graphene underneath \emph{PDOT:PSS}. So far, no devices containing that same layer of graphene could be assembled, but it is hoped that they will show an according increase in conversion efficiency. 
\section*{Conclusion}
In conclusion, optimal \emph{PDOT:PSS} properties could be identified, their optimum on glass directly translating to optimal devices. %I think the conclusions should be more open-ended, instead of saying that the optimal properties were identified, say something about how improved device performance correlated to improved transmission and resistivity\\ 
The side project involving graphene shows promising preliminary results and will be pursued further in this project's remaining time with the hope to ultimately produce high quality decives.
\end{document}

% I read the website, and it looked like the restrictions about sections were just for the talk.  So, unless they specifically told you to have section titles for the abstract, I think you can get rid of them. Also, I feel like it needs a sentence somewhere explaining the structure of the solar cell, and why optimal transmissivity and optimal conductivity of the PEDOT are desirable.
