\documentclass[a4paper,10pt]{article}
\usepackage[english]{babel}
\usepackage[hmargin=2.5cm,vmargin=3.5cm]{geometry}
\usepackage{microtype}
\pagestyle{empty}

\begin{document}

\section*{Shottky-type solar cells}
Schottky-type solar cells offer the possibility to attain light to electricity conversion efficiencies similar to ``standard'' solar cells, without the need for high-energy, high-cost \emph{p-n}-junction assembly from different types of silicon (\emph{Si}). The ultimate efficiency of the Schottky-type solar cell is largely dependant on the properties of the polymer film and its interface with \emph{Si}. Therefore, it is the aim of this project to optimise poly(ethylene-dioxythiophene):poly(styrene-sulfonate) (\emph{PEDOT:PSS}) film properties.
\section*{Optimising surfactant concentration}
This project's focus is on the optimal amount of fluoro-surfactant -- here Zonyl\textsuperscript{\textregistered} -- to be used in watery \emph{PDOT:PSS} solution also containing 5 \%\footnote{all weight-percentages are with respect to \emph{PDOT:PSS}} ethylene-glycol. Surfactant concentration has been varied in a set of experiments. The different solutions were spun-cast on glass plates, then analysed with four-point probe resistance measurements, transmission spectrometry as well as atomic force and optical microscopy. Solar cell devices were assembled from these films and \emph{n}-type-\emph{Si} in a lift-off float-on (\emph{LOFO}) process. Silver top contacts were created using electron beam evaporation and the devices' performance was tested with standard J-V measurements.
\section*{Experimental results}
On glass, maximum conductivity as well as maximum transmissivity was found for films containing 0.4\% Zonyl. Solar cell devices assembled using the \emph{LOFO}-process and films containing 0.4\% Zonyl showed the highest efficiency.\\
In a side project, the film's resistivity could be reduced by a factor of $\approx \, 4$ by introducing a thin film of graphene underneath \emph{PDOT:PSS}. So far, no devices containing that same layer of graphene could be assembled, but it is hoped that they will show an according increase in conversion efficiency. 
\section*{Conclusion}
In conclusion, optimal \emph{PDOT:PSS} properties could be identified, their optimum on glass directly translating to optimal devices.\\
The side project involving graphene shows promising preliminary results and will be pursued further in this project's remaining time with the hope to ultimately produce high quality devices.
\end{document}
