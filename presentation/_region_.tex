\message{ !name(abstract.tex)}\documentclass[a4paper]{article}
\usepackage[english]{babel}
\usepackage[hmargin=3cm,vmargin=3.5cm]{geometry}
\usepackage{microtype}
\pagestyle{empty}

\begin{document}

\message{ !name(abstract.tex) !offset(-3) }


\section{Can we improve inversion layer Si-cells}
Schottky-type solar cells offer the possibility to attain light to electricity conversion efficiencies similar to ``standard'' solar cells. Since they can be produced from n\footnote{or p}-type silicone (Si) and polymer films only, they are producible at a fraction of the cost of their standard counter parts and can incorporate solution-based fabricatio techniques.  
\section{Optimising surfactant concentration}
The ultimate efficiency of the Schottky-type solar cell is largely dependant on the quality of the polymer film which is used to induce the inversion layer within the Si-base. Therefore it is the aim of this project to optimise film properties to the end of maximising the device's quality.
\section{Results}
In a set of experiments, the surfactant concentration in the polymer solution has been varied while all other parameters were held constant. The resultant films were analysed with four-point probe resistance measurements, transmission spectrometry and atomic force microscopy (AFM) to find their thickness. Optimum, i.e. minimal, resistance was found for film with a surfacant concentration of 0.5 weight percent with respect to PDOT:PSS. Solar cell devices assembled in a lift-off-float-on (LOFO) process from films containing 0.5\% Zonyl accordingly showed the highest efficiency.\n
In a side project, the film's resistivity could be reduced by a factor of $\approx \, 7$ by introducing a thin\footnote{3 to 8 atomic layers} film of graphene underneath. So far, no devices containing that same layer of graphene could be assembled, but it is hoped that they will show an according increase in coversion efficiency. 
\section{Conclusion}
In conclusion, PDOT:PSS properties similar to what has been found in the literature could be reproduced, their optimum directly translating to optimal devices.\n
The side project involving graphene shows promising preliminary results and will be pursued further in this project's the remaining time with the hope to ultimately produce high quality decives.
\end{document}

\message{ !name(abstract.tex) !offset(-24) }
